\section{QGIS}
QGIS adalah perangkat Sistem Informasi Geografis (SIG) Open Source yang user friendly dengan lisensi di bawah GNU General Public License. QGIS merupakan proyek tidak resmi dari Open Source Geospatial Foundation (OSGeo). QGIS dapat dijalankan pada Linux, Unix, Mac OSX, Windows dan Android, serta mendukung banyak format dan fungsionalitas data vektor, raster, dan basisdata.
Quantum GIS mendukung penggunaan "GPS tools" untuk menggunggah (upload) atau mengunduh (download) data langsung ke unit GPS. Pengguna juga dapat mengkonversi format-format GPS ke format GPX atau melakukan import dan export terhadap data format GPX yang ada.
Andaikan pengguna memiliki sebuah web server yang telah terpasang fitur UMN MapServer, pengguna dapat menpublikasi map di internet untuk berbagi (sharing) dengan pengguna lainnya.


\subsection{Getting QGIS}
\begin{enumerate}
\item
Buka browser internet Anda dan ketikkan pada bagian atas jendela browser Anda dengan tulisan qgis.org. Kemudian tekan Enter.
\begin{figure}[ht]
	    \centerline{\includegraphics[width=0.50\textwidth]{figures/image1}}
	    \end{figure}

Situs resmi QGIS akan terlihat seperti ini:
\begin{figure}[ht]
	    \centerline{\includegraphics[width=1.00\textwidth]{figures/image2}}
	    \end{figure}
\item
Klik Unduh Sekarang
\item
Jika Anda menggunakan Windows, klik pada QGIS Standalone Installer Version 2.8 (32 bit). Nomor versi komputer Anda mungkin berbeda.
\begin{figure}[ht]
	    \centerline{\includegraphics[width=0.50\textwidth]{figures/image3}}
	    \end{figure}
\item
Jika Anda tidak menggunakan Windows, pilih Sistem Operasi Anda dari menu yang tersedia. Ikuti intruksi instalasi.
\begin{figure}[ht]
	    \centerline{\includegraphics[width=1.00\textwidth]{figures/image4}}
	    \end{figure}
\item
Ketika file selesai didownload, jalankan dan ikuti perintah untuk menginstal QGIS.

\end{enumerate}