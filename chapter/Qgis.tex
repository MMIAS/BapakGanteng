\section{QGIS}
QGIS adalah perangkat Sistem Informasi Geografis (SIG) Open Source yang user friendly dengan lisensi di bawah GNU General Public License. QGIS merupakan proyek tidak resmi dari Open Source Geospatial Foundation (OSGeo). QGIS dapat dijalankan pada Linux, Unix, Mac OSX, Windows dan Android, serta mendukung banyak format dan fungsionalitas data vektor, raster, dan basisdata.
Quantum GIS mendukung penggunaan "GPS tools" untuk menggunggah (upload) atau mengunduh (download) data langsung ke unit GPS. Pengguna juga dapat mengkonversi format-format GPS ke format GPX atau melakukan import dan export terhadap data format GPX yang ada.
Andaikan pengguna memiliki sebuah web server yang telah terpasang fitur UMN MapServer, pengguna dapat menpublikasi map di internet untuk berbagi (sharing) dengan pengguna lainnya.

Sistem Informasi Geografis (SIG) adalah sebuah sistem yang dirancang untuk memungkinkan orang-orang bekerja dengan data yang berkaitan dengan suatu tempat di bumi. SIG memungkinkan pembuatan, penyimpanan, manipulasi, dan analisis data geografis. SIG merupakan konsep yang sangat luas dan dapat melibatkan perangkat keras dan perangkat lunak yang rumit. Tetapi, untuk memenuhi tujuan kebanyakan orang, yang dibutuhkan adalah sebuah perangkat lunak SIG yang sederhana, dan pada unit ini kita akan mempelajari bagaimana menggunakan aplikasi open-source yang unggul, QGIS.

\subsection{Getting QGIS}
\begin{enumerate}
\item
Buka browser internet Anda dan ketikkan pada bagian atas jendela browser Anda dengan tulisan qgis.org. Kemudian tekan Enter.\ref{image1}.
\begin{figure}[ht]
        \centerline{\includegraphics[width=0.25\textwidth]{figures/image1}}
        \caption{gambar}
        \label{image1}
        \end{figure}

Situs resmi QGIS akan terlihat seperti ini:\ref{image2}.
\begin{figure}[ht]
        \centerline{\includegraphics[width=1.00\textwidth]{figures/image2}}
        \caption{gambar}
        \label{image2}
        \end{figure}
\item
Klik Unduh Sekarang
\item
Jika Anda menggunakan Windows, klik pada QGIS Standalone Installer Version 2.8 (32 bit). Nomor versi komputer Anda mungkin berbeda.\ref{image3}.
\begin{figure}[ht]
        \centerline{\includegraphics[width=0.25\textwidth]{figures/image3}}
        \caption{gambar}
        \label{image3}
        \end{figure}
\item
Jika Anda tidak menggunakan Windows, pilih Sistem Operasi Anda dari menu yang tersedia. Ikuti intruksi instalasi.\ref{image4}.
\begin{figure}[ht]
        \centerline{\includegraphics[width=0.25\textwidth]{figures/image4}}
        \caption{gambar}
        \label{image4}
        \end{figure}
\item
Ketika file selesai didownload, jalankan dan ikuti perintah untuk menginstal QGIS.
\end{enumerate}

\subsection{Installing QGIS}
\begin{enumerate}
\item
Buka folder dimana anda menyimpan file instalasi QGIS.\ref{image5}.
\begin{figure}[ht]
        \centerline{\includegraphics[width=0.05\textwidth]{figures/image5}}
        \caption{gambar}
        \label{image5}
        \end{figure}
\item
Jalankan file instalasi tersebut. Jika Anda menginstal QGIS versi 2.x, akan terlihat seperti ini:\ref{image6}.
\begin{figure}[ht]
        \centerline{\includegraphics[width=0.25\textwidth]{figures/image6}}
        \caption{gambar}
        \label{image6}
        \end{figure}
\item 
Klik Next
\item
Klik I Agree untuk setuju dengan syarat dan ketentuan yang berlaku.\ref{image7}.
\begin{figure}[ht]
        \centerline{\includegraphics[width=0.25\textwidth]{figures/image7}}
        \caption{gambar}
        \label{image7}
        \end{figure}
\item
Pada jendela berikutnya Anda akan ditanyakan dimana Anda akan menginstall QGIS. Pada kasus umum, pengaturan bawaan yang ada sudah dapat digunakan. Klik Next.\ref{image8}.
\begin{figure}[ht]
        \centerline{\includegraphics[width=0.25\textwidth]{figures/image8}}
        \caption{gambar}
        \label{image8}
        \end{figure}
\item
Pada jendela berikutnya, klik Install tanpa mencentang apapun yang ada di dalam kotak.\ref{image9}.
\begin{figure}[ht]
        \centerline{\includegraphics[width=0.25\textwidth]{figures/image9}}
        \caption{gambar}
        \label{image9}
        \end{figure}

QGIS akan memulai untuk menginstall. Ini mungkin akan membutuhkan beberapa waktu untuk selesai.\ref{image10}.
\begin{figure}[ht]
        \centerline{\includegraphics[width=0.25\textwidth]{figures/image10}}
        \caption{gambar}
        \label{image10}
        \end{figure}
\item
Klik Finish untuk melengkapi instalasi. Kemudian komputer Anda akan me-reboot secara otomatis.\ref{image11}.
\begin{figure}[ht]
        \centerline{\includegraphics[width=0.25\textwidth]{figures/image11}}
        \caption{gambar}
        \label{image11}
        \end{figure}
\item
Buka QGIS dari Start Menu, berikut tampilan QGIS.\ref{image12}.
\begin{figure}[ht]
        \centerline{\includegraphics[width=0.25\textwidth]{figures/image12}}
        \caption{gambar}
        \label{image12}
        \end{figure}
\end{enumerate}

\subsection{Classification}
Pemberian label adalah cara yang baik untuk mengkomunikasikan informasi seperti nama dari suatu tempat, namun label tidak dapat digunakan untuk semua hal. Sebagai contoh, misalnya kita ingin menunjukkan tiap jenis vegetasi tersebut berada di kecamatan mana. Dengan menggunakan label, akan tampak seperti ini:\ref{image13}.
\begin{figure}[ht]
        \centerline{\includegraphics[width=0.25\textwidth]{figures/image13}}
        \caption{gambar}
        \label{image13}
        \end{figure}
Bisa kita lihat, hal tersebut tampak tidak ideal, jadi kita membutuhkan solusi yang lain. Itulah yang akan kita pelajari pada bagian ini! Pada bagian ini, kita akan mempelajari bagaimana melakukan klasifikasi data vektor dengan efektif.


\subsection{Toolbar}
Pada bagian atas dari tampilan QGIS terdapat banyak sekali tool, dimana masing-masing tool tersebut masuk ke dalam beberapa kategori “toolbar”. Sebagai contoh,  file mengizinkan Anda untuk menyimpan, membuka, mencetak dan memulai proyek baru. Kita telah menggunakan salah satu alat dari toolbar file ketika kita membuka proyek baru.\ref{toolbar}.
\begin{figure}[ht]
    \centerline{\includegraphics[width=0.25\textwidth]{figures/toolbar.png}}
    \caption{gambar toolbar yang ada pada QGis}
    \label{toolbar}
    \end{figure}

Dengan menggerakkan mouse ke atas ikon, nama tool akan muncul untuk membantu mengidentifikasi setiap tool. Jumlah tool yang ada (tombol) akan tampak sangat banyak pada awalnya, tapi Anda perlahan akan mengenalnya. Tool yang ada dikelompokkan sesuai dengan fungsi pada toolbars. Jika Anda melihat lebih dekat, Anda akan melihat titik-titik vertikal sejumlah sepuluh titik pada bagian kiri dari setiap toolbar. Jika Anda meng-klik dan menahannya dengan menggunakan mouse ,dapat menggerakkan toolbar ke tempat yang lebih sesuai atau memisahkannnya sesuai.\ref{toolbar1}.
\begin{figure}[ht]
    \centerline{\includegraphics[width=0.25\textwidth]{figures/toolbar1.png}}
    \caption{gambar toolbar yang ada pada QGis}
    \label{toolbar1}
    \end{figure}
    
\subsection{Status Bar}
Status bar akan menampilkan informasi mengenai peta Anda. Ini juga memperbolehkan Anda untuk mengatur skala peta dan melihat koordinat yang mouse Anda arahkan pada peta.

Koordinat peta ini sama dengan tipe koordinat yang telah Anda pelajari ketika Anda belajar mengenai GPS. Status bar ini akan menunjukkan posisi lintang dan bujur dari kursor mouse Anda.
Mungkin hal ini masih belum terlalu jelas untuk Anda, tapi seiring dengan meningkatnya kemampuan Anda di SIG, hal ini nantinya akan terlihat masuk akal. \ref{statbar}.
\begin{figure}[ht]
    \centerline{\includegraphics[width=0.25\textwidth]{figures/statbar.png}}
    \caption{gambar status bar yang ada pada QGis}
    \label{statbar}
    \end{figure}

\subsection{Menu Bar}
Menu bar memberikan akses ke berbagai fitur QGIS menggunakan hirarki menu standar. Menu-menu utama dan ringkasan dari beberapa pilihan menu yang tercantum di bawah ini, bersama dengan ikon alat yang sesuai seperti yang ditampilkan pada toolbar, serta cara pintas keyboard. Cara pintas keyboard juga dapat dikonfigurasi secara manual (cara pintas yang disajikan dalam bagian ini adalah default), menggunakan [Configure Shortcuts] alat di bawah Setting. Beberapa pilihan menu hanya muncul jika plugin yang sesuai dimuat. Untuk informasi lebih lanjut tentang alat dan toolbar, lihat Bagian Toolbar. \ref{menubar}
\begin{enumerate}
\item
Project
\begin{figure}[ht]
    \centerline{\includegraphics[width=0.25\textwidth]{figures/menubar.png}}
    \caption{Gambar project pada Menu Bar}
    \label{menubar}
    \end{figure}
\item
Edit
\begin{figure}[ht]
    \centerline{\includegraphics[width=0.25\textwidth]{figures/menubar1.png}}
    \caption{Gambar edit pada Menu Bar}
    \label{menubar1}
    \end{figure}
\item
View
\begin{figure}[ht]
    \centerline{\includegraphics[width=0.25\textwidth]{figures/menubar2.png}}
    \caption{Gambar view pada Menu Bar}
    \label{menubar2}
    \end{figure}
\item
Setting
\begin{figure}[ht]
    \centerline{\includegraphics[width=0.25\textwidth]{figures/menubar3.png}}
    \caption{Gambar setting pada Menu Bar}
    \label{menubar3}
    \end{figure}
\item
Plugin
\begin{figure}[ht]
    \centerline{\includegraphics[width=0.25\textwidth]{figures/menubar4.png}}
    \caption{Gambar plugin pada Menu Bar}
    \label{menubar4}
    \end{figure}
\item
Vector
\begin{figure}[ht]
    \centerline{\includegraphics[width=0.25\textwidth]{figures/menubar5.png}}
    \caption{Gambar vector pada Menu Bar}
    \label{menubar5}
    \end{figure}
\item
Raster
\begin{figure}[ht]
    \centerline{\includegraphics[width=0.25\textwidth]{figures/menubar6.png}}
    \caption{Gambar raster pada Menu Bar}
    \label{menubar6}
    \end{figure}
\item
Database
\begin{figure}[ht]
    \centerline{\includegraphics[width=0.25\textwidth]{figures/menubar7.png}}
    \caption{Gambar database pada Menu Bar}
    \label{menubar7}
    \end{figure}
\item
Pengolahan
\begin{figure}[ht]
    \centerline{\includegraphics[width=0.25\textwidth]{figures/menubar8.png}}
    \caption{Gambar pengolahan pada Menu Bar}
    \label{menubar8}
    \end{figure}
\item
Bantuan
\begin{figure}[ht]
    \centerline{\includegraphics[width=0.25\textwidth]{figures/menubar9.png}}
    \caption{Gambar bantuan pada Menu Bar}
    \label{menubar9}
    \end{figure}
\end{enumerate}

\subsection{Atribut}
Data gis mempunyai dua bagian - fitur dan attribut. Atributt adalah data terstruktur mengenai setiap fitur. Tutorial ini menunjukkan bagaimana memperlihatkan attribut dan melakukan query standard pada attribut di QGIS.\ref{atribut}
Berikut langkahnya :
\begin{enumerate}
\item
Buka proyek yang telah dikerjakan sebelumnya
Pada proyek yang kita kerjakan ada lokasi, jalur kereta, dan beberapa jalan, tetapi kita tidak dapat melihat seluruh data yang terkandung dalam layer-layer tersebut.
\item
Pilih salah satu jalan pada panel daftar Layer.
\item
Klik kanan, dan klik tombol Open Attribute Table :
Akann terlihat tabel dengan data yang lebih banyak tentang layer jalan. Data ekstra tersebut disebut data atribut. Garis-garis yang Anda dapat lihat pada peta Anda menggambarkan kemana garis tersebut menuju – ini merupakan data spatial. Anda akan mengingat pada JOSM dimana terdapat bagian yang sama. Titik, garis, dan bentuk yang Anda gambar memberitahu kita dimana, tetapi tags, atau atribut, memberitahukan kita apa
\item
Lihatlah pada tabel atribut. Setiap baris tabel menghubungkan satu fitur pada layer jalan. Setiap kolom mengandung satu atribut. Jika Anda memilih layer lain dan mengklik tombol Open Attribute Table, maka Anda akan melihat tabel yang berbeda
\item
Tutup tabel atribut
\end{enumerate}

\subsection{Label Tool}
Label dapat ditambahkan ke dalam peta untuk menunjukkan informasi tentang objek. Suatu layer vektor dapat memiliki label yang berkaitan dengan layer tersebut. Label bergantung pada data atribut dari layer terkait.
Ada beberapa cara untuk menambahkan label pada QGIS, tetapi beberapa cara lebih baik dibandingkan yang lain. Dapat dilihat ketika membuka jendela properti dari sebuah layer, terdapat tabs yang bertuliskan "Labels". Walaupun tab ini dirancang untuk memberikan label pada peta yang dibuat, tab ini fungsinya tidak sebaik "Tool Label", yang akan dipelajari pada bagian ini. Sebelum mengakses Tool Label, harus dipastikan bahwa fitur tersebut telah diaktifkan. Berikut langkah-langkahnya :\ref{label}
\begin{enumerate}
\item
Pergi ke menu item View --> Toolbars
\item
Pastikan item Label telah memiliki tanda centang disebelahnya. Jika belum, klik pada item Label dan fitur tersebut akan diaktifkan.\ref{label}.
\begin{figure}[ht]
    \centerline{\includegraphics[width=0.25\textwidth]{figures/label.PNG}}
    \caption{Toolbar Label tampak seperti ini}
    \label{label}
    \end{figure}
\item
Klik layer POI\_Bandung\_OSM yang terdapat pada panel Daftar Layer, sehingga layer tersebut tersorot
\item
Klik tombol Layer Labelling Options
\begin{figure}[ht]
    \centerline{\includegraphics[width=0.25\textwidth]{figures/layer.PNG}}
    \caption{tombol Layer Labelling Options}
    \label{layer}
    \end{figure}
\item
Setelah klik tombol diatas maka akan muncul halaman pengaturan Layer Labelling. Centang kotak yang ada tulisan Layer this Label With
\begin{figure}[ht]
    \centerline{\includegraphics[width=0.25\textwidth]{figures/laylabel.PNG}}
    \caption{tombol Layer this Label With}
    \label{laylabel}
    \end{figure}
\item
Pilih Field Name untuk pemberian label
\begin{figure}[ht]
    \centerline{\includegraphics[width=0.25\textwidth]{figures/name.PNG}}
    \caption{Field Name}
    \label{name}
    \end{figure}
\item
Klik OK maka peta akan muncul dengan label
\end{enumerate}

\subsection{Fitur Dasar Quantum GIS Dalam Pengelolaan Data Vektor dan Raster}
Sebagai perangkat lunak Sistem Informasi Geografik, Quantum GIS memiliki kapabilitas untuk menampilkan, mengolah dan menyajikan data. Secara garis besar, Quantum GIS memiliki kemampuan sebagai berikut:
\begin{enumerate}
\item
Membaca dan mengedit data dalam format vektor dan raster, termasuk data atribut
Quantum GIS dapat membaca dan mengolah data dalam banyak format, baik dalam bentuk raster maupun vektor. Sebagian diantara format data yang bisa diolah Quantum GIS, didukung oleh library GDAL dan OGR. Beberapa format data vektor yang umum digunakan dan bisa diolah Quantum GIS, antara lain: Shapefile, MapInfo Table, S-57, KML, AutoCAD DXF, dll.
Fitur editing peta vektor, antara lain split, merge/union, vertex editing, pemindahan/drag posisi objek peta, dan sebagainya. Dalam hal editing data raster, fitur dasar Quantum GIS antara lain bisa digunakan untuk clipping, pembuatan kontur, interpolasi grid dan sebagainya
\item
Konversi sistem koordinat dan proyeksi peta
Konversi sistem koordinat dan proyeksi peta dapat dilakukan dengan mudah di Quantum GIS, dengan memilih opsi CRS ketika menyimpan data, seperti gambar berikut:
\begin{figure}[ht]
    \centerline{\includegraphics[width=0.25\textwidth]{figures/proyeksi.png}}
    \caption{Gambar konversi sistem koordinat dan proyeksi peta pada QGIS}
    \label{proyeksi}
    \end{figure}
\item
Navigasi peta
Pada Quantum GIS, navigasi peta bisa dilakukan melalui toolbar khusus, dengan fungsi navigasi yang bisa digunakan antara lain: Perbesar (zoom in), Perkecil (zoom out), Penggeseran (pan), Zoom ke layer yang aktif/terpilih, Zoom ke objek yang terseleksi, Zoom ke seluruh tampilan peta, Zoom previous/next, untuk kembali ke tampilan sebelumnya atau sesudahnya.
\item
Setting tampilan peta
Tampilan peta dapat diatur melalui menu Layer > Properties. Hal-hal yang dapat diatur antara lain: warna dan pola arsiran, warna dan ketebalan garis, bentuk dan ukuran simbol, dan sebagainya. Tampilan setting layer seperti berikut:
\begin{figure}[ht]
    \centerline{\includegraphics[width=0.25\textwidth]{figures/setting.png}}
    \caption{Gambar setting tampilan peta pada QGIS}
    \label{setting}
    \end{figure}
\end{enumerate}

\subsection{Menambahkan Data Vektor}
\begin{enumerate}
\item 
Buka proyek QGIS yang baru. Peta dan daftar layer yang akan tampak masih kosong.
\item
Terdapat dua cara untuk menambahkan sebuah layer vektor baru pada proyek. pertama dapat mengarahkan ke menu :menuselection:` Layer > Tambah Lapisan > Tambahkan Layer Vektor...` atau kedua dapat klik pada tombol Tambahkan Layer Vektor pada toolbar.
\item
Jika Anda tidak dapat menemukan tombol toolbar, klik kanan pada toolbar dan pastikan kotak toolbar Mengatur Layer sudah tercentang.
\item
Klik pada tombol Navigasi dan Telusuri untuk data latihan Anda. Pergilah ke direktori qgis/Sleman/ dan pilih Jalan\_Sleman\_OSM, POI\_Sleman dan Kecamatan\_Sleman. Anda dapat memilih lebih dari satu file dengan menahan tombol CTRL pada keyboard dan klik tiap file. Klik Open lalu Open lagi.

Peta Anda sekarang akan terlihat seperti ini:\ref{vektor}
\begin{enumerate}
\item
Project
\begin{figure}[ht]
    \centerline{\includegraphics[width=0.25\textwidth]{figures/vektor.PNG}}
    \caption{Gambar vektor pada qgis}
    \label{vektor}
    \end{figure}

\end{enumerate}

\subsection{Mode Algoritma}
Mode ini menggunakan algoritma yang berbeda untuk memilah data ke kelas-kelas yang terpisah.
•	Interval sama: seperti namanya, metode ini akan menghasilkan kelas dengan ukuran yang sama. Jika data kita terentang dari 0-100 dan kita ingin 10 kelas, metode ini akan menghasilkan sebuah kelas dari 0-10, 10-20, 20-30 dan seterusnya, kelas-kelas mempunyai ukuran yang sama dari 10 unit.
•	Kuantil - Metode ini akan memutuskan kelas di mana jumlah nilai pada setiap kelas adalah sama. Jika terdapat 100 nilai dan kita ingin 4 kelas, metode kuantil akan menentukkan setiap kelas akan mempunyai nilai 25.

\subsection{Klasifikasi Data Vektor}
Dapat menampilkan simbol yang berbeda pada tiap objek, tergantung atributnya, contoh: Peta pariwisata, isinya objek yang telah diklasifikasikan berdasar jenis tempat wisata: 
seperti tempat wisata sejarah ataupun tempat wisata belanja.



